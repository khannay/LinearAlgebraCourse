\documentclass[10pt, a4paper]{article}
%\usepackage{geometry}
\usepackage[inner=2.5cm,outer=2.5cm,top=2.5cm,bottom=2.5cm]{geometry}
\pagestyle{empty}
\usepackage{graphicx}
\usepackage{fancyhdr, lastpage, bbding, pmboxdraw}
\usepackage[usenames,dvipsnames]{color}
\usepackage[colorlinks,pagebackref,pdfusetitle,urlcolor=darkblue,citecolor=darkblue,linkcolor=darkred,bookmarksnumbered,plainpages=false]{hyperref}
\usepackage{amsfonts}
\usepackage{listings}
\usepackage{caption}
\usepackage{placeins}
\DeclareCaptionFont{white}{\color{white}}
\DeclareCaptionFormat{listing}{\colorbox{gray}{\parbox{\textwidth}{#1#2#3}}}
\captionsetup[lstlisting]{format=listing,labelfont=white,textfont=white}
\usepackage{verbatim} % used to display code
\usepackage{fancyvrb}
\usepackage{acronym}
\usepackage{amsthm}
\usepackage{amsmath}
\usepackage{pgfplots}
\VerbatimFootnotes


%Set up the headers
\pagestyle{fancyplain}
\fancyhf{}
\lhead{ \fancyplain{}{\classNum} }
\chead{ \fancyplain{}{\topic} }
\rhead{ \fancyplain{}{Dr. Hannay} }
%\rfoot{\fancyplain{}{page \thepage\ of \pageref{LastPage}}}
\fancyfoot[RO, LE] {page \thepage\ of \pageref{LastPage} }
\thispagestyle{plain}


%Set up the theorem enviroment
\newtheoremstyle{break}
  {\topsep}{\topsep}%
  {\itshape}{}%
  {\bfseries}{}%
  {\newline}{}%
\theoremstyle{break}

\newtheorem{theorem}{Theorem}
\newtheorem{lemma}[theorem]{Lemma}
\newtheorem{conj}[theorem]{Conjecture}
\newtheorem{defn}[theorem]{\underline{Definition:}}

%Set up custom commands
\renewcommand\thesection{\Roman{section}.}
%exercise command \ex{question}{space times 0.5 inches}
\newcommand{\question}[3]     {
 #1 (\textbf{#2 points}) \newline 
\newcount\Scount
\Scount=0
\loop\vspace*{0.5in}\par\goodbreak\advance\Scount by 1 \ifnum\Scount< #3 \repeat
}%Makes an exercise with space left for the students to write the answer
\newcommand{\makesol}[1]     {
\newcount\Scount
\Scount=0
\loop\vspace*{0.5in}\par\goodbreak\advance\Scount by 1 \ifnum\Scount< #1 \repeat
}%Makes an exercise with space left for the students to write the answer

\newcommand{\fpquestion}[3]     {
 #1 (\textbf{#2 points}) \newline 
\newpage
}%Makes an exercise with space left for the students to write the answer

%Calculus specific shorthands
\newcommand{\dlim}{\lim_{h\rightarrow0}}



%Make ShortHands for quick updates
\renewcommand{\thefootnote}{\fnsymbol{footnote}}
\newcommand{\topic}{{Exam IV}}
\newcommand{\classNum}{{MATH 3324}}
\newcommand{\prob}{{\mathbb{P}}}

\definecolor{OliveGreen}{cmyk}{0.64,0,0.95,0.40}
\definecolor{CadetBlue}{cmyk}{0.62,0.57,0.23,0}
\definecolor{lightlightgray}{gray}{0.93}
\definecolor{darkblue}{rgb}{0,0,.6}
\definecolor{darkred}{rgb}{.7,0,0}
\definecolor{darkgreen}{rgb}{0,.6,0}
\definecolor{red}{rgb}{.98,0,0}


\lstset{
%language=bash,                          % Code langugage
basicstyle=\ttfamily,                   % Code font, Examples: \footnotesize, \ttfamily
keywordstyle=\color{OliveGreen},        % Keywords font ('*' = uppercase)
commentstyle=\color{gray},              % Comments font
numbers=left,                           % Line nums position
numberstyle=\tiny,                      % Line-numbers fonts
stepnumber=1,                           % Step between two line-numbers
numbersep=5pt,                          % How far are line-numbers from code
backgroundcolor=\color{lightlightgray}, % Choose background color
frame=none,                             % A frame around the code
tabsize=2,                              % Default tab size
captionpos=t,                           % Caption-position = bottom
breaklines=true,                        % Automatic line breaking?
breakatwhitespace=false,                % Automatic breaks only at whitespace?
showspaces=false,                       % Dont make spaces visible
showtabs=false,                         % Dont make tabls visible
columns=flexible,                       % Column format
morekeywords={__global__, __device__},  % CUDA specific keywords
}

%%%%%%%%%%%%%%%%%%%%%%%%%%%%%%%%%%%%
\begin{document}
\begin{center}
{\large \textsc{\underline{MATH 3312: Quiz I}}}
\vspace{0.125in}
\end{center}
\underline{Directions:} \\
Complete this exam without consulting with \underline{anyone} else for help answering the questions (electronically, in person, etc). You may use your notes, the notes on S1 and your textbook. Additionally, you are allowed to use numerical computing software to aid you in answering the questions. \textbf{You must show all of your work. If you use numerical software then write in the command you used. Failure to show all of the steps used to obtain your solution will result in NO points being awarded for that question. } 
\noindent The quiz must be turned in by the beginning of class on Weds 2/7/2017. If you want to turn it in early make sure you hand it to me in person (don't put it in the box outside my office). Good luck! \\
\noindent\rule{15cm}{0.4pt}

\vspace{0.25in}

\begin{enumerate}
\item (15 points) Create a $3 \times 3$ matrix equation $Ax=b$ which will give no solutions (pick a matrix $A$ and a right-hand side $b$). Prove this by performing a Gaussian elimination on it. Make sure you show your steps (you can check this in Matlab, but show the steps). 
\item (15 points) Alter your choice of the right-hand side $b$ for the first question so that it now has infinite solutions. Demonstrate this using a Gaussian elimination. 
\item (20 points) Start with the column vector $$u_0=\begin{bmatrix} 0 \\ 1 \end{bmatrix}$$ Multiply again and again by the \textbf{Markov matrix} $$A= \begin{bmatrix}
0.5 & 0.20 \\
0.5 & 0.80
\end{bmatrix} $$
so that $u_n=A^n u_0$. For example $u_1$ is given by:
$$u_1=\begin{bmatrix}
0.5 & 0.20 \\
0.5 & 0.80
\end{bmatrix} \begin{bmatrix} 0 \\1 \end{bmatrix}$$
Find $u_1,u_2,u_3$. What property do you notice about the $u_0, u_1, u_2, ...$ vectors? For $n$ very large what does $u_n$ become?
\item (10 points) What elimination matrix $E_{21}$ is required to reduce the following matrix to row echelon form $U$? (ready for solution by back substitution).
\begin{equation}
E_{21}
\begin{bmatrix}
1 & 1 \\
2 & -1 
\end{bmatrix}
=\begin{bmatrix}
1 &1 \\ 0 & ?
\end{bmatrix}
\end{equation}
\item (15 points) Write the following matrix as a sum of \textit{outer products}. 
\begin{equation}
\begin{bmatrix}
 1 & 2 & 0 \\
 0 & 3 & 0 \\
 -1 & 0 &0
\end{bmatrix}
\end{equation}
\item (15 points) Suppose you solve $Ax=b$ for the three special right hand sides $b$:
\begin{align}
& Ax_1=
\begin{bmatrix}
1 \\
0 \\
0\\
\end{bmatrix}
& Ax_2=
\begin{bmatrix}
0 \\
1 \\
0\\
\end{bmatrix}
& & Ax_3=
\begin{bmatrix}
0 \\
0 \\
1 \\
\end{bmatrix}
\end{align}
If the three solutions $x_1,x_2,x_3$ are the columns of a matrix $X$ then what is $AX$?
\item (10 points) Find all solutions to the following system $Ax=b$:
\begin{equation}
\begin{bmatrix}
8 & 1 & 6 \\
3&5& 7\\
4 & 9 &2 \\
\end{bmatrix}
x=
\begin{bmatrix}
4 \\
6 \\
20 \\
\end{bmatrix}
\end{equation}
\end{enumerate}





\end{document}