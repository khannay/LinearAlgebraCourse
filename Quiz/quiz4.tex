\documentclass[10pt, a4paper]{article}
%\usepackage{geometry}
\usepackage[inner=2.5cm,outer=2.5cm,top=2.5cm,bottom=2.5cm]{geometry}
\pagestyle{empty}
\usepackage{graphicx}
\usepackage{fancyhdr, lastpage, bbding, pmboxdraw}
\usepackage[usenames,dvipsnames]{color}
\usepackage[colorlinks,pagebackref,pdfusetitle,urlcolor=darkblue,citecolor=darkblue,linkcolor=darkred,bookmarksnumbered,plainpages=false]{hyperref}
\usepackage{amsfonts}
\usepackage{listings}
\usepackage{caption}
\usepackage{placeins}
\DeclareCaptionFont{white}{\color{white}}
\DeclareCaptionFormat{listing}{\colorbox{gray}{\parbox{\textwidth}{#1#2#3}}}
\captionsetup[lstlisting]{format=listing,labelfont=white,textfont=white}
\usepackage{verbatim} % used to display code
\usepackage{fancyvrb}
\usepackage{acronym}
\usepackage{amsthm}
\usepackage{amsmath}
\usepackage{pgfplots}
\VerbatimFootnotes


%Set up the headers
\pagestyle{fancyplain}
\fancyhf{}
\lhead{ \fancyplain{}{\classNum} }
\chead{ \fancyplain{}{\topic} }
\rhead{ \fancyplain{}{Dr. Hannay} }
%\rfoot{\fancyplain{}{page \thepage\ of \pageref{LastPage}}}
\fancyfoot[RO, LE] {page \thepage\ of \pageref{LastPage} }
\thispagestyle{plain}


%Set up the theorem enviroment
\newtheoremstyle{break}
  {\topsep}{\topsep}%
  {\itshape}{}%
  {\bfseries}{}%
  {\newline}{}%
\theoremstyle{break}

\newtheorem{theorem}{Theorem}
\newtheorem{lemma}[theorem]{Lemma}
\newtheorem{conj}[theorem]{Conjecture}
\newtheorem{defn}[theorem]{\underline{Definition:}}

%Set up custom commands
\renewcommand\thesection{\Roman{section}.}
%exercise command \ex{question}{space times 0.5 inches}
\newcommand{\question}[3]     {
 #1 (\textbf{#2 points}) \newline 
\newcount\Scount
\Scount=0
\loop\vspace*{0.5in}\par\goodbreak\advance\Scount by 1 \ifnum\Scount< #3 \repeat
}%Makes an exercise with space left for the students to write the answer
\newcommand{\makesol}[1]     {
\newcount\Scount
\Scount=0
\loop\vspace*{0.5in}\par\goodbreak\advance\Scount by 1 \ifnum\Scount< #1 \repeat
}%Makes an exercise with space left for the students to write the answer

\newcommand{\fpquestion}[3]     {
 #1 (\textbf{#2 points}) \newline 
\newpage
}%Makes an exercise with space left for the students to write the answer

%Calculus specific shorthands
\newcommand{\dlim}{\lim_{h\rightarrow0}}



%Make ShortHands for quick updates
\renewcommand{\thefootnote}{\fnsymbol{footnote}}
\newcommand{\topic}{{Quiz III}}
\newcommand{\classNum}{{MATH 3312}}
\newcommand{\prob}{{\mathbb{P}}}

\definecolor{OliveGreen}{cmyk}{0.64,0,0.95,0.40}
\definecolor{CadetBlue}{cmyk}{0.62,0.57,0.23,0}
\definecolor{lightlightgray}{gray}{0.93}
\definecolor{darkblue}{rgb}{0,0,.6}
\definecolor{darkred}{rgb}{.7,0,0}
\definecolor{darkgreen}{rgb}{0,.6,0}
\definecolor{red}{rgb}{.98,0,0}


\lstset{
%language=bash,                          % Code langugage
basicstyle=\ttfamily,                   % Code font, Examples: \footnotesize, \ttfamily
keywordstyle=\color{OliveGreen},        % Keywords font ('*' = uppercase)
commentstyle=\color{gray},              % Comments font
numbers=left,                           % Line nums position
numberstyle=\tiny,                      % Line-numbers fonts
stepnumber=1,                           % Step between two line-numbers
numbersep=5pt,                          % How far are line-numbers from code
backgroundcolor=\color{lightlightgray}, % Choose background color
frame=none,                             % A frame around the code
tabsize=2,                              % Default tab size
captionpos=t,                           % Caption-position = bottom
breaklines=true,                        % Automatic line breaking?
breakatwhitespace=false,                % Automatic breaks only at whitespace?
showspaces=false,                       % Dont make spaces visible
showtabs=false,                         % Dont make tabls visible
columns=flexible,                       % Column format
morekeywords={__global__, __device__},  % CUDA specific keywords
}

%%%%%%%%%%%%%%%%%%%%%%%%%%%%%%%%%%%%
\begin{document}
\begin{center}
{\large \textsc{\underline{MATH 3312: Quiz IV}}}
\vspace{0.125in}
\end{center}
\underline{Directions:} \\
Complete this quiz without consulting with \underline{anyone} else for help answering the questions (electronically, in person, etc). You may use your notes, the notes on Canvas and your textbook. Additionally, you are allowed to use numerical computing software to aid you in answering the questions. \textbf{You must show all of your work. If you use numerical software then write in the command you used. Failure to show all of the steps used to obtain your solution will result in NO points being awarded for that question. }  \\
\noindent\rule{15cm}{0.4pt}

\vspace{0.25in}

\begin{enumerate}
\item (25 points) Find the eigenvalues and eigenvectors for the following matrix
\begin{equation}
A=
\begin{bmatrix}
1 & -2 \\
-1 &0 
\end{bmatrix}
\nonumber
\end{equation}

%Answer:  The eigenvalues of this matrix are $\lambda_1=-1, \lambda_2=2.$ with the eigenvectors
%\begin{equation}
%v_1=
%\begin{bmatrix}
%1 \\ 
%1
%\end{bmatrix}
%\qquad
%v_2=
%\begin{bmatrix}
%-2 \\
%1
%\end{bmatrix} \nonumber
%\end{equation}






\item (25 points) For the below \textbf{Markov Matrix} M, we have that $x_{t}=M^tx_0$ find the long-time solution $x_{\infty}$. You will want to use Julia/Matlab (or something else) to help
find the eigenvalues/eigenvectors.
\begin{equation}
M=
\begin{bmatrix}
0.1 & 0.05 &0.20 \\
0.5 & 0.10 & 0.50 \\
0.40 & 0.85 & 0.30 
\end{bmatrix}
\nonumber
\end{equation}

%Answer: The equilibrium will be the eigenvector with eigenvalue $1$. Using matlab we find that $v_1=[ -0.20,-0.56,-0.80]^T$ to convert this to a probability value we need to take $x_{\infty}=v_1/sum(v_1)$, which gives:
\begin{equation}
x_{\infty}=
\begin{bmatrix}
0.13312 \\
0.35714 \\
0.50974
\end{bmatrix}
\nonumber
\end{equation}




\item (25 points)  Let $v_1$ be an eigenvector of a $n \times n$ matrix $A$ with eigenvalue $\lambda_1$. If the matrix $B$ is defined as $B=A+I_n$ where $I_n$ is the identity matrix.  Show that $v_1$ is also an eigenvector of $B$ and find its eigenvalue. 

%Answer: $(B+I)v_1=Bv_1+v_1=\lambda_1 v_1+v_1=(\lambda+1) v_1$


\item (25 points) Perform a Gram-Schmidt reduction on the vectors $a=[1,1]^T$ and $b=[2,1]^T$ to find $q_1$ and $q_2$ such that $q_1 \cdot q_2=<q_1|q_2>=0$ and $||q_1||=||q_2||=1$. 

%Answer: $q_1=[1/\sqrt{2}, 1/\sqrt{2}]^T$ and $$q_2=\left[I-\frac{|a><a|}{(||a||^2)}\right]b=[1/\sqrt{2},-1/\sqrt{2}]^T$$ after we normalize the result. 



\end{enumerate}

























\end{document}
