\documentclass[12pt, a4paper]{article}
%\usepackage{geometry}
\usepackage[inner=2.5cm,outer=2.5cm,top=2.5cm,bottom=2.5cm]{geometry}
\pagestyle{empty}
\usepackage{graphicx}
\usepackage{fancyhdr, lastpage, bbding, pmboxdraw}
\usepackage[usenames,dvipsnames]{color}
\usepackage[colorlinks,pagebackref,pdfusetitle,urlcolor=darkblue,citecolor=darkblue,linkcolor=darkred,bookmarksnumbered,plainpages=false]{hyperref}
\usepackage{amsfonts}
\usepackage{listings}
\usepackage{caption}
\usepackage{placeins}
\DeclareCaptionFont{white}{\color{white}}
\DeclareCaptionFormat{listing}{\colorbox{gray}{\parbox{\textwidth}{#1#2#3}}}
\captionsetup[lstlisting]{format=listing,labelfont=white,textfont=white}
\usepackage{verbatim} % used to display code
\usepackage{fancyvrb}
\usepackage{acronym}
\usepackage{amsthm}
\usepackage{amsmath}
\usepackage{pgfplots}
\VerbatimFootnotes

%For using units 
\usepackage{siunitx}

%Packages for drawing triangles
\usepackage{tkz-euclide}
\usetkzobj{all}
%Packages to add highlighting
\usepackage{xcolor}
\usepackage{soul}


%Set up the headers
\pagestyle{fancyplain}
\fancyhf{}
\lhead{ \fancyplain{}{\classNum} }
\chead{ \fancyplain{}{\topic} }
\rhead{ \fancyplain{}{Dr. Hannay} }
%\rfoot{\fancyplain{}{page \thepage\ of \pageref{LastPage}}}
\fancyfoot[RO, LE] {page \thepage\ of \pageref{LastPage} }
\thispagestyle{plain}


%Set up the theorem enviroment
\newtheoremstyle{break}
  {\topsep}{\topsep}%
  {\itshape}{}%
  {\bfseries}{}%
  {\newline}{}%
\theoremstyle{break}

\newtheorem{theorem}{Theorem}
\newtheorem{lemma}[theorem]{Lemma}
\newtheorem{conj}[theorem]{Conjecture}
\newtheorem{defn}[theorem]{\underline{Definition:}}

%Set up custom commands
\renewcommand\thesection{\Roman{section}.}
%exercise command \ex{question}{space times 0.5 inches}
\newcommand{\question}[3]     {
 #1 (\textbf{#2 points}) \newline 
\newcount\Scount
\Scount=0
\loop\vspace*{0.5in}\par\goodbreak\advance\Scount by 1 \ifnum\Scount< #3 \repeat
}%Makes an exercise with space left for the students to write the answer
\newcommand{\makesol}[1]     {
\newcount\Scount
\Scount=0
\loop\vspace*{0.5in}\par\goodbreak\advance\Scount by 1 \ifnum\Scount< #1 \repeat
}%Makes an exercise with space left for the students to write the answer

\newcommand{\fpquestion}[3]     {
 #1 (\textbf{#2 points}) \newline 
\newpage
}%Makes an exercise with space left for the students to write the answer

%Calculus specific shorthands
\newcommand{\dlim}{\lim_{h\rightarrow0}}
\newcommand{\myint}{\displaystyle \int}

%Exams only option for math displays
\everymath{\displaystyle}

%Make ShortHands for quick updates
\renewcommand{\thefootnote}{\fnsymbol{footnote}}
\newcommand{\topic}{{Midterm}}
\newcommand{\classNum}{{MATH 3312}}
\newcommand{\prob}{{\mathbb{P}}}

\definecolor{OliveGreen}{cmyk}{0.64,0,0.95,0.40}
\definecolor{CadetBlue}{cmyk}{0.62,0.57,0.23,0}
\definecolor{lightlightgray}{gray}{0.93}
\definecolor{darkblue}{rgb}{0,0,.6}
\definecolor{darkred}{rgb}{.7,0,0}
\definecolor{darkgreen}{rgb}{0,.6,0}
\definecolor{red}{rgb}{.98,0,0}

%%%%%%%%%%%%%%%%%%%%%%%%%%%%%%%%%%%%
\begin{document}
\begin{flushleft}
\Large \textbf{Name: } \underline{\hspace{8cm}} \\
\end{flushleft}
\vspace{0.25in}
\begin{center}
{\Large \textsc{\underline{MATH 3312: Midterm Exam}}}
\vspace{0.125in}
\end{center}
\section{True/False Questions}
Are the following statements true or false? Each question is worth 2 points. 
\begin{enumerate}
\item \underline{\hspace{1cm}} The matrix $
\begin{bmatrix}
-2 & 1 & 0 \\
0 &6 &-10 \\
0 & 0 & -2 
\end{bmatrix} $ is invertible
\item \underline{\hspace{1cm}}  If $A$ is an $m\times n$ matrix of rank $r=m<n$ then $Ax=b$ has infinite solutions for any $b$. 
\item \underline{\hspace{1cm}} If the matrix A has a trivial nullspace then it has full column rank.
\item \underline{\hspace{1cm}} If $C$ is a $6 \times 10$ matrix then it is a function $C: \mathbb{R}^6 \longrightarrow \mathbb{R}^{10}$
\item \underline{\hspace{1cm}} Let $A$ be a matrix and $R$ be the reduced row echelon form of $A$. Then $C(A^T)=C(R^T)$.  
\item \underline{\hspace{1cm}} A square matrix $A$ is invertible if and only if the nullspace $N(A)$ contains just the zero vector $\vec{0}$. 
\item  \underline{\hspace{1cm}} A matrix $A$ where $N(A) \neq \vec{0}$ cannot have a unique solution to any $Ax=b$. 
\item  \underline{\hspace{1cm}} If a matrix $C$ is $10 \times 3$ then the system must have either 0 or infinite solutions to a system $Cx=b$. 
\item  \underline{\hspace{1cm}} A square matrix can have no free variables when row reduced.
\item  \underline{\hspace{1cm}} If $G$ is a $4 \times 5$ matrix with rank $4$ then the dimension of the nullspace is $0$. 
\end{enumerate}
\newpage


\section{Questions}
\begin{enumerate}
\item (20  points) Solve the following system for $\vec{x}=\langle x_1,x_2,x_3 \rangle$
\begin{equation}
\begin{bmatrix}
1&2&-1 \\
0&1&1 \\
2&0&3 \\
\end{bmatrix}
\begin{bmatrix}
x_1\\
x_2 \\
x_3 
\end{bmatrix}=
\begin{bmatrix}
-4\\
2\\
11
\end{bmatrix} \nonumber
\end{equation} 
%Ans=[1 -1 3]
\newpage
\item (20 points) Find the LU decomposition of the following matrix
\begin{equation}
A=
\begin{bmatrix}
2 & 1 & 0 \\
-4 & -1 & 4 \\
2 & 1 & 2 \\
\end{bmatrix}
\end{equation}

%Ans L=[1 0 0; -2 1 0; 1 0 1]
%U=[2 1 0; 0 1 4; 0 0 2]
\newpage
\item (20 points)  Find the complete solution (general solution) to the following system $Ax=b$:
\begin{equation}
\begin{bmatrix}
1 & 3 & 1 &2  \\
2&6 &4 & 8 \\
0&0&2&4 \\
\end{bmatrix}
\begin{bmatrix}
x_1\\
x_2 \\
x_3 \\
x_4 
\end{bmatrix}=
\begin{bmatrix}
1 \\
4 \\
2 
\end{bmatrix} \nonumber
\end{equation}
You may be interested to know that the reduced row echelon form of the matrix above is:
$$
\begin{bmatrix}
1 & 3 & 0 &0  \\
0&0 &1 & 2 \\
0&0&0&0 \\
\end{bmatrix} \nonumber
$$
\newpage

\item (20 points) Find the nullspace of the below matrix
\begin{equation}
\begin{bmatrix}
1 & 2 & 1 \\
3&6&3 \\
-1&-2&-1
\end{bmatrix} \nonumber
\end{equation}
You may be interested to know that the Reduced Row Echelon Form (RREF) of this matrix is given by:
\begin{equation}
\begin{bmatrix}
1 & 2 & 1 \\
0&0&0 \\
0&0&0
\end{bmatrix} \nonumber
\end{equation}
%ans [1,3,-1][1,2,1]
\newpage







\end{enumerate}

\end{document}