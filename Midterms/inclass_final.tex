\documentclass[12pt, a4paper]{article}
%\usepackage{geometry}
\usepackage[inner=2.5cm,outer=2.5cm,top=2.5cm,bottom=2.5cm]{geometry}
\pagestyle{empty}
\usepackage{graphicx}
\usepackage{fancyhdr, lastpage, bbding, pmboxdraw}
\usepackage[usenames,dvipsnames]{color}
\usepackage[colorlinks,pagebackref,pdfusetitle,urlcolor=darkblue,citecolor=darkblue,linkcolor=darkred,bookmarksnumbered,plainpages=false]{hyperref}
\usepackage{amsfonts}
\usepackage{listings}
\usepackage{caption}
\usepackage{placeins}
\DeclareCaptionFont{white}{\color{white}}
\DeclareCaptionFormat{listing}{\colorbox{gray}{\parbox{\textwidth}{#1#2#3}}}
\captionsetup[lstlisting]{format=listing,labelfont=white,textfont=white}
\usepackage{verbatim} % used to display code
\usepackage{fancyvrb}
\usepackage{acronym}
\usepackage{amsthm}
\usepackage{amsmath}
\usepackage{pgfplots}
\VerbatimFootnotes

%For using units 
\usepackage{siunitx}

%Packages for drawing triangles
\usepackage{tkz-euclide}
\usetkzobj{all}
%Packages to add highlighting
\usepackage{xcolor}
\usepackage{soul}


%Set up the headers
\pagestyle{fancyplain}
\fancyhf{}
\lhead{ \fancyplain{}{\classNum} }
\chead{ \fancyplain{}{\topic} }
\rhead{ \fancyplain{}{Dr. Hannay} }
%\rfoot{\fancyplain{}{page \thepage\ of \pageref{LastPage}}}
\fancyfoot[RO, LE] {page \thepage\ of \pageref{LastPage} }
\thispagestyle{plain}


%Set up the theorem enviroment
\newtheoremstyle{break}
  {\topsep}{\topsep}%
  {\itshape}{}%
  {\bfseries}{}%
  {\newline}{}%
\theoremstyle{break}

\newtheorem{theorem}{Theorem}
\newtheorem{lemma}[theorem]{Lemma}
\newtheorem{conj}[theorem]{Conjecture}
\newtheorem{defn}[theorem]{\underline{Definition:}}

%Set up custom commands
\renewcommand\thesection{\Roman{section}.}
%exercise command \ex{question}{space times 0.5 inches}
\newcommand{\question}[3]     {
 #1 (\textbf{#2 points}) \newline 
\newcount\Scount
\Scount=0
\loop\vspace*{0.5in}\par\goodbreak\advance\Scount by 1 \ifnum\Scount< #3 \repeat
}%Makes an exercise with space left for the students to write the answer
\newcommand{\makesol}[1]     {
\newcount\Scount
\Scount=0
\loop\vspace*{0.5in}\par\goodbreak\advance\Scount by 1 \ifnum\Scount< #1 \repeat
}%Makes an exercise with space left for the students to write the answer

\newcommand{\fpquestion}[3]     {
 #1 (\textbf{#2 points}) \newline 
\newpage
}%Makes an exercise with space left for the students to write the answer

%Calculus specific shorthands
\newcommand{\dlim}{\lim_{h\rightarrow0}}
\newcommand{\myint}{\displaystyle \int}

%Exams only option for math displays
\everymath{\displaystyle}

%Make ShortHands for quick updates
\renewcommand{\thefootnote}{\fnsymbol{footnote}}
\newcommand{\topic}{{Final}}
\newcommand{\classNum}{{MATH 3312}}
\newcommand{\prob}{{\mathbb{P}}}

\definecolor{OliveGreen}{cmyk}{0.64,0,0.95,0.40}
\definecolor{CadetBlue}{cmyk}{0.62,0.57,0.23,0}
\definecolor{lightlightgray}{gray}{0.93}
\definecolor{darkblue}{rgb}{0,0,.6}
\definecolor{darkred}{rgb}{.7,0,0}
\definecolor{darkgreen}{rgb}{0,.6,0}
\definecolor{red}{rgb}{.98,0,0}

%%%%%%%%%%%%%%%%%%%%%%%%%%%%%%%%%%%%
\begin{document}
\begin{flushleft}
\Large \textbf{Name: } \underline{\hspace{8cm}} \\
\end{flushleft}
\vspace{0.25in}
\begin{center}
{\Large \textsc{\underline{MATH 3312: Final Exam}}}
\vspace{0.125in}
\end{center}
Some of these questions may look familiar from previous work. However, be wary because they \textbf{may have changed slightly}. You may have your textbook and (1) notebook of notes. \underline{No calculators are allowed!}
\section{True/False Questions}
Are the following statements true or false? Each question is worth 2 points. 
\begin{enumerate}
\item \underline{\hspace{1cm}} If a $3 \times 3$ matrix has the eigenvalues $\{ -1,1,10\}$ then it is invertible. 
\item \underline{\hspace{1cm}}  If $A$ is an $m\times n$ matrix of rank $r=n<m$ then $Ax=b$ has infinite solutions for any $b$. 
\item \underline{\hspace{1cm}} A projection matrix $P$ has all zeros and ones as eigenvalues. 
\item \underline{\hspace{1cm}} Let $v$ be in the row space of $A$ and $n$ be in the nullspace of $A$. Then $v \cdot n=\langle v | n \rangle=0$
\item \underline{\hspace{1cm}} A set of distinct vectors $v_1, v_2, v_3, v_4$ can form a basis for $\mathbb{R}^3$. 
\item \underline{\hspace{1cm}} If $R$ is the reduced row echelon form of the matrix $A$ then $C(A)=C(R)$. 
\item  \underline{\hspace{1cm}} If $A$ is an $m \times n$ matrix then the dimension of the row space must be less than $n$. 
\item  \underline{\hspace{1cm}} A linear system $Ax=b$ must have either zero, one or infinite solutions. 
\item  \underline{\hspace{1cm}} If a matrix has ONLY $\vec{0}$ in its nullspace then it is invertible.
\item  \underline{\hspace{1cm}} The matrix $
\begin{bmatrix}
-2 & 1 \\
1 & 3 \\
\end{bmatrix}$
has real eigenvalues.
\end{enumerate}
\newpage


\section{Questions}
\begin{enumerate}
\item (10  points) Solve the following system for $\vec{x}=\langle x_1,x_2,x_3 \rangle$
\begin{equation}
\begin{bmatrix}
1 & 3 &4  \\
1 & 0 & 1 \\
0 & 0 & 2 \\
\end{bmatrix}
\begin{bmatrix}
x_1\\
x_2 \\
x_3 
\end{bmatrix}=
\begin{bmatrix}
6 \\
3 \\
4
\end{bmatrix} \nonumber
\end{equation} 
Ans=[1,-1,2]
\newpage
\item (10 points)  Find the eigenvalues and eigenvectors for the following matrix:
\begin{equation}
\begin{bmatrix}
1 & 0 \\
4 & -1
\end{bmatrix}
\nonumber
\end{equation}
%Ans -1, 1 [0,1], v2=[1,2]
\newpage

\item (10 points) Find the least squares solution to the following system:
\begin{equation}
\begin{bmatrix}
1 & 1 \\ 
2 & 4 \\
0 & 0 \\
\end{bmatrix}
\begin{bmatrix}
x_1 \\
x_2 \\
x_3 
\end{bmatrix}
=
\begin{bmatrix}
1 \\
1 \\
1 
\end{bmatrix} \nonumber
\end{equation}
%Ans=[1.5, -0.5]
\newpage

\item (10 points) Find the dimensions of the \textbf{four fundamental subspaces} for the following matrix.
%Change Numbers
\begin{equation}
A=\begin{bmatrix}
3 & 4 &-1&0 & 0\\
0 & 1 & 0 & 1 & 0\\
6 & 8 &-2 &1& 0 \\
\end{bmatrix} \nonumber
\end{equation}
\newpage

\item (10 points)  Is the following set of vectors a basis for $\mathbb{R}^3$? Why or why not? Show your work. 
\begin{align}
& v_1=\begin{bmatrix} 1 \\ -1 \\ 1 \end{bmatrix}, \qquad v_2=\begin{bmatrix} 1 \\ 2 \\ 3 \end{bmatrix} , \qquad v_3=\begin{bmatrix} -1 \\ -5\\-5 \end{bmatrix} \nonumber
\end{align}
%Ans. No [1,-2,1] is in the nullspace of this matrix
\newpage

\item (10 points) What are the \textbf{\underline{eigenvalues}}  of the following matrix? 
\begin{equation}
A=
\begin{bmatrix}
1 & 2 & 3 \\
0  & 3 & 4 \\
0 & 0 & 10 
\end{bmatrix} \nonumber
\end{equation}
\textit{Hint: I haven't broken my promise. This can be done without any computation at all. }
\newpage

\item (10 points) Find a basis for the nullspace of the following matrix:
\begin{equation}
A=
\begin{bmatrix}
2 & 0 & -1 \\
-2 & 1 & 4 \\
4& 0 & -2  
\end{bmatrix} \nonumber
\end{equation}
%Ans[1/2, -3.0,1]
\newpage

\item (10 points) Prove that if $\vec{x} \in \mathbb{R}^n$ and $\vec{y}\in \mathbb{R}^n$ are orthogonal vectors (i.e $\vec{x} \cdot \vec{y}=0$) and $Q$ is an $n \times n$ orthonormal matrix then  $(Q\vec{x}) \cdot (Q\vec{y})=0$. 
\newpage








\end{enumerate}

\end{document}