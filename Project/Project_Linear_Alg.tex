\documentclass[11pt, a4paper]{article}
%\usepackage{geometry}
\usepackage[inner=1.5cm,outer=1.5cm,top=2.5cm,bottom=2.5cm]{geometry}
\pagestyle{empty}
\usepackage{graphicx}
\usepackage{fancyhdr, lastpage, bbding, pmboxdraw}
\usepackage[usenames,dvipsnames]{color}
\definecolor{darkblue}{rgb}{0,0,.6}
\definecolor{darkred}{rgb}{.7,0,0}
\definecolor{darkgreen}{rgb}{0,.6,0}
\definecolor{red}{rgb}{.98,0,0}
\usepackage[colorlinks,pagebackref,pdfusetitle,urlcolor=darkblue,citecolor=darkblue,linkcolor=darkred,bookmarksnumbered,plainpages=false]{hyperref}
\usepackage{amsfonts}

%Make ShortHands for quick updates
\renewcommand{\thefootnote}{\fnsymbol{footnote}}
\newcommand{\sems}{{Fall 2017}}
\newcommand{\classNum}{{MATH 2330}}
\newcommand{\prob}{{\mathbb{P}}}


\pagestyle{fancyplain}
\fancyhf{}
\lhead{ \fancyplain{}{\classNum} }
\chead{ \fancyplain{}{\sems} }
\rhead{ \fancyplain{}{Dr. Hannay} }
%\rfoot{\fancyplain{}{page \thepage\ of \pageref{LastPage}}}
\fancyfoot[RO, LE] {page \thepage\ of \pageref{LastPage} }
\thispagestyle{plain}

%%%%%%%%%%%% LISTING %%%
\usepackage{listings}
\usepackage{caption}
\DeclareCaptionFont{white}{\color{white}}
\DeclareCaptionFormat{listing}{\colorbox{gray}{\parbox{\textwidth}{#1#2#3}}}
\captionsetup[lstlisting]{format=listing,labelfont=white,textfont=white}
\usepackage{verbatim} % used to display code
\usepackage{fancyvrb}
\usepackage{acronym}
\usepackage{amsthm}
\VerbatimFootnotes % Required, otherwise verbatim does not work in footnotes!




\definecolor{OliveGreen}{cmyk}{0.64,0,0.95,0.40}
\definecolor{CadetBlue}{cmyk}{0.62,0.57,0.23,0}
\definecolor{lightlightgray}{gray}{0.93}



\lstset{
%language=bash,                          % Code langugage
basicstyle=\ttfamily,                   % Code font, Examples: \footnotesize, \ttfamily
keywordstyle=\color{OliveGreen},        % Keywords font ('*' = uppercase)
commentstyle=\color{gray},              % Comments font
numbers=left,                           % Line nums position
numberstyle=\tiny,                      % Line-numbers fonts
stepnumber=1,                           % Step between two line-numbers
numbersep=5pt,                          % How far are line-numbers from code
backgroundcolor=\color{lightlightgray}, % Choose background color
frame=none,                             % A frame around the code
tabsize=2,                              % Default tab size
captionpos=t,                           % Caption-position = bottom
breaklines=true,                        % Automatic line breaking?
breakatwhitespace=false,                % Automatic breaks only at whitespace?
showspaces=false,                       % Dont make spaces visible
showtabs=false,                         % Dont make tabls visible
columns=flexible,                       % Column format
morekeywords={__global__, __device__},  % CUDA specific keywords
}

%%%%%%%%%%%%%%%%%%%%%%%%%%%%%%%%%%%%
\begin{document}
\begin{center}
{\Large \textsc{MATH 3312:  Computational Project}}
\end{center}
%\date{September 26, 2014}

\begin{center}
\rule{6.5in}{0.4pt}
\begin{minipage}[t]{0.90 \textwidth}
\textbf{Topic:}  Discrete Random Variables \hspace{2.0in} \textbf{Due Date:}  5/8\\
\end{minipage}
\rule{6.5in}{0.4pt}
\end{center}
\vspace{.5cm}
\setlength{\unitlength}{1in}
\renewcommand{\arraystretch}{2}

\section*{Directions:}


\section*{Questions:}
\begin{enumerate}
\item \underline{Markov Chains:} The fundamental dogma of molecular biology is said to be that DNA is transcribed into RNA and the RNA is used to produce proteins. Proteins are the active form of genes and can be used to catalyze enzymatic reactions within the cell. 

\item \underline{Spectral Clustering:}


\item \underline{Image Processing: }

\item \underline{Random Matrices:} We may generate random matrices in Matlab using the command randn(n,n). Find the eigenvalues of a random $1000 \times 1000$ matrix $R$ and make a scatter plot of the eigenvalues in the complex plane. That is plot the real part of each eigenvalue on the x axis and the imaginary part on the y axis. 
\begin{enumerate}
\item What shape do the eigenvalues form in the complex plane? 
\item Does this shape change when you plot a different random matrix?
\item Generate a random network using the command $N=double(randn(1000,1000)<0.1);$. Make a plot of the eigenvalues of this network? Does the spectrum of this matrix take the same shape?
\end{enumerate}



\end{enumerate}


%%%%%% THE END 
\end{document} 